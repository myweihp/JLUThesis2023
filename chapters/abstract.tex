\jluCAbstract{
   电子计算机(亦稱电脑)是利用数字电子技术,根据一系列指令指示並且自动执行任意算术或逻辑操作序列的设备。通用计算机因有能遵循被称为“程序”的一般操作集的能力而使得它们能够执行极其广泛的任务。

计算机被用作各种工业和消费设备的控制系统。这包括简单的特定用途设备(如微波炉和遥控器)、工业设备(如工业机器人和计算机辅助设计),及通用设备(如个人电脑和智能手机之类的移动设备)等。尽管计算机种类繁多,但根据图灵机理论,一部具有著基本功能的计算机,应当能够完成任何其它计算机能做的事情。因此,理论上从智能手机到超级计算机都应该可以完成同样的作业(不考虑时间和存储因素)。由于科技的飞速进步,下一代计算机总是在性能上能够显著地超过其前一代,这一现象有时被称作“摩尔定律”。通过互联网,计算机互相连接,极大地提高了信息交换速度,反过来推动了科技的发展。在21世纪的现在,计算机的应用已经涉及到方方面面,各行各业了。

自古以来,简单的手动设备——就像算盘——帮助人们进行计算。在工业革命初期,各式各样机械的出现,而初衷都是为了自动完成冗长而乏味的任务,例如织机的编织图案。更复杂的机器在20世纪初出现,通过模拟电路进行复杂特定的计算。第一台数字电子计算机出现于二战期间。自那时以来,电脑的速度,功耗和多功能性則不断增加。在现代,机械计算机的应用已经完全被电子计算机所取代。

计算机在组成上形式不一,早期计算机的天依然有大量体积庞大的巨型计算机为特别的科学计算或面向大型组织的事务处理需求服务。比较小的,为个人应用而设计的称为微型计算机(Personal Computer,PC),在中國地區简称為「微机」。我們今天在日常使用“计算机”一词时通常也是指此,不过现在计算机最为普遍的应用形式却是嵌入式,嵌入式计算机通常相对简单、体积小,并被用来控制其它设备——无论是飞机、工业机器人还是数码相机。

同计算机相关的技术研究叫電腦科學,而「计算机技术」指的是将计算机科学的成果应用于工程实践所派生的诸多技术性和经验性成果的总合。「计算机技术」与「计算机科学」是两个相关而又不同的概念,它们的不同在于前者偏重于实践而后者偏重于理论。至於由数据为核心的研究則称為信息技术。

传统上,现代计算机包含至少一个处理单元(通常是中央处理器(CPU))和某种形式的存储器。处理元件执行算术和逻辑运算,并且排序和控制单元可以响应于存储的信息改变操作的顺序。外围设备包括输入设备(键盘,鼠标,操纵杆等)、输出设备(显示器屏幕,打印机等)以及执行两种功能(例如触摸屏)的输入/输出设备。外围设备允许从外部来源检索信息,并使操作结果得以保存和检索。

电子计算机(亦稱电脑)是利用数字电子技术,根据一系列指令指示並且自动执行任意算术或逻辑操作序列的设备。通用计算机因有能遵循被称为“程序”的一般操作集的能力而使得它们能够执行极其广泛的任务。

计算机被用作各种工业和消费设备的控制系统。这包括简单的特定用途设备(如微波炉和遥控器)、工业设备(如工业机器人和计算机辅助设计),及通用设备(如个人电脑和智能手机之类的移动设备)等。尽管计算机种类繁多,但根据图灵机理论,一部具有著基本功能的计算机,应当能够完成任何其它计算机能做的事情。因此,理论上从智能手机到超级计算机都应该可以完成同样的作业(不考虑时间和存储因素)。由于科技的飞速进步,下一代计算机总是在性能上能够显著地超过其前一代,这一现象有时被称作“摩尔定律”。通过互联网,计算机互相连接,极大地提高了信息交换速度,反过来推动了科技的发展。在21世纪的现在,计算机的应用已经涉及到方方面面,各行各业了。

自古以来,简单的手动设备——就像算盘——帮助人们进行计算。在工业革命初期,各式各样机械的出现,而初衷都是为了自动完成冗长而乏味的任务,例如织机的编织图案。更复杂的机器在20世纪初出现,通过模拟电路进行复杂特定的计算。第一台数字电子计算机出现于二战期间。自那时以来,电脑的速度,功耗和多功能性則不断增加。在现代,机械计算机的应用已经完全被电子计算机所取代。

计算机在组成上形式不一,早期计算机的天依然有大量体积庞大的巨型计算机为特别的科学计算或面向大型组织的事务处理需求服务。比较小的,为个人应用而设计的称为微型计算机(Personal Computer,PC),在中國地區简称為「微机」。我們今天在日常使用“计算机”一词时通常也是指此,不过现在计算机最为普遍的应用形式却是嵌入式,嵌入式计算机通常相对简单、体积小,并被用来控制其它设备——无论是飞机、工业机器人还是数码相机。

同计算机相关的技术研究叫電腦科學,而「计算机技术」指的是将计算机科学的成果应用于工程实践所派生的诸多技术性和经验性成果的总合。「计算机技术」与「计算机科学」是两个相关而又不同的概念,它们的不同在于前者偏重于实践而后者偏重于理论。至於由数据为核心的研究則称為信息技术。

传统上,现代计算机包含至少一个处理单元(通常是中央处理器(CPU))和某种形式的存储器。处理元件执行算术和逻辑运算,并且排序和控制单元可以响应于存储的信息改变操作的顺序。外围设备包括输入设备(键盘,鼠标,操纵杆等)、输出设备(显示器屏幕,打印机等)以及执行两种功能(例如触摸屏)的输入/输出设备。外围设备允许从外部来源检索信息,并使操作结果得以保存和检索。
}

\jluCKeywords{甲, 乙, 丙}

\jluEAbstract{
   A computer is a machine that can be instructed to carry out sequences of arithmetic or logical operations automatically via computer programming. Modern computers have the ability to follow generalized sets of operations, called programs. These programs enable computers to perform an extremely wide range of tasks. A ``complete'' computer including the hardware, the operating system (main software), and peripheral equipment required and used for ``full'' operation can be referred to as a computer system. This term may as well be used for a group of computers that are connected and work together, in particular a computer network or computer cluster.

Computers are used as control systems for a wide variety of industrial and consumer devices. This includes simple special purpose devices like microwave ovens and remote controls, factory devices such as industrial robots and computer-aided design, and also general purpose devices like personal computers and mobile devices such as smartphones. The Internet is run on computers and it connects hundreds of millions of other computers and their users.

Early computers were only conceived as calculating devices. Since ancient times, simple manual devices like the abacus aided people in doing calculations. Early in the Industrial Revolution, some mechanical devices were built to automate long tedious tasks, such as guiding patterns for looms. More sophisticated electrical machines did specialized analog calculations in the early 20th century. The first digital electronic calculating machines were developed during World War II. The first semiconductor transistors in the late 1940s were followed by the silicon-based MOSFET (MOS transistor) and monolithic integrated circuit (IC) chip technologies in the late 1950s, leading to the microprocessor and the microcomputer revolution in the 1970s. The speed, power and versatility of computers have been increasing dramatically ever since then, with MOS transistor counts increasing at a rapid pace (as predicted by Moore's law), leading to the Digital Revolution during the late 20th to early 21st centuries.

Conventionally, a modern computer consists of at least one processing element, typically a central processing unit (CPU) in the form of a metal-oxide-semiconductor (MOS) microprocessor, along with some type of computer memory, typically MOS semiconductor memory chips. The processing element carries out arithmetic and logical operations, and a sequencing and control unit can change the order of operations in response to stored information. Peripheral devices include input devices (keyboards, mice, joystick, etc.), output devices (monitor screens, printers, etc.), and input/output devices that perform both functions (e.g., the 2000s-era touchscreen). Peripheral devices allow information to be retrieved from an external source and they enable the result of operations to be saved and retrieved.
}

\jluEKeywords{a, b, c}