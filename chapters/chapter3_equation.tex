\chapter{公式的使用}
\label{chap:equation}

\section{带点的行间公式}

在学校给出的模板中,公式后面是有一排点的,原作者给的dotsequation环境的点似是靠下的,修改后是上下局中的,与模板一致。这里给出了多行公示使用同一个编号的例子,注意换行:

\begin{dotsequation}
    \label{eq:spe}
    \begin{aligned}
            \mathrm{SPE}(pos, 2\times idx) &=\mathrm{sin}(\frac{pos}{10000^{2\times idx/C}} ),\\
    \mathrm{SPE}(pos, 2\times idx +1) &= \mathrm{cos}(\frac{pos}{10000^{2\times idx/C}} ),
    \end{aligned}
\end{dotsequation}%

\section{行内公式}
行内公示非常简单,用\$...\$包裹即可。
样例:

其中$\ell=1...L$,$L$为Transformer层的数量。
